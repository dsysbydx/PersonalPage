\documentclass[10pt]{amsart}
\usepackage{geometry}                % See geometry.pdf to learn the layout options. There are lots.
\geometry{letterpaper}                   % ... or a4paper or a5paper or ... 
%\geometry{landscape}                % Activate for for rotated page geometry
%\usepackage[parfill]{parskip}    % Activate to begin paragraphs with an empty line rather than an indent
\usepackage{fullpage} %to reduce margin
\usepackage{graphicx}
\usepackage{amssymb}
\usepackage{amsthm}
\usepackage{amsmath}
\usepackage{epstopdf}
\usepackage{color}
\usepackage{framed} % or, "mdframed"
%\usepackage[framed]{ntheorem}
%\newframedtheorem{theo}{Theorem}
\newtheorem{thm}{Theorem}
\newtheorem{example}{Example}


\DeclareGraphicsRule{.tif}{png}{.png}{`convert #1 `dirname #1`/`basename #1 .tif`.png}


\newcommand{\bi}{\begin{itemize}}
\renewcommand{\i}{\item}
\newcommand{\ei}{\end{itemize}}
\renewcommand{\ni}{\noindent}
\newcommand{\tf}{\textbf}
\newcommand{\ti}{\textit}
\newcommand{\ld}{\lambda}
\renewcommand{\l}{\left}
\renewcommand{\r}{\right}
\newcommand{\la}{\langle}
\newcommand{\ra}{\rangle}
\newcommand{\tsg}{\tilde{\sigma}}
\newcommand{\sg}{\sigma}
\newcommand{\mc}{\mathcal}
\newcommand{\R}{\mathbb{R}}
\newcommand{\Rd}{\mathbb{R}^{d}}
\newcommand{\Rtd}{\mathbb{R}^{2d}}
\newcommand{\eps}{\varepsilon}

\newcommand{\cA}{\mathcal{A}}
\newcommand{\cH}{\mathcal{H}}
\newcommand{\cP}{\mathcal{P}}
\newcommand{\cL}{\mathcal{L}}
\newcommand{\PoR}{\mathcal{P}_{1}(\mathbb{R})}
\newcommand{\tdW}{d\tilde{W}}
\newcommand{\tZ}{\tilde{Z}}
\newcommand{\halpha}{\hat{\alpha}}
\newcommand{\hX}{\hat{X}}
\newcommand{\hY}{\hat{Y}}
\newcommand{\hZ}{\hat{Z}}
\newcommand{\htZ}{\hat{\tilde{Z}}}
\newcommand{\hm}{\hat{m}}
\renewcommand{\d}[1]{\partial_{#1}}
\newcommand{\Ve}{V^{\eps}}
\newcommand{\ue}{u^{\eps}}
\newcommand{\me}{m^{\eps}}
\newcommand{\Xie}{X^{i,\eps}}
\newcommand{\aie}{\alpha^{i,\eps}}
\renewcommand{\ae}{\alpha^{\eps}}
\newcommand{\Vo}{V^{0}}
\newcommand{\uo}{u^{0}}
\newcommand{\mo}{m^{0}}
\newcommand{\Xio}{X^{i,0}}
\newcommand{\aio}{\alpha^{i,0}}
\newcommand{\ao}{\alpha^{0}}
\newcommand{\du}{ \delta^{u,\eps}}
\newcommand{\dm}{ \delta^{m,\eps}}
\newcommand{\duo}{ \delta^{u}}
\newcommand{\dmo}{ \delta^{m}}
\newcommand{\mC}{\mathcal}
\newcommand{\imply}{ \quad \Rightarrow \quad}
\newcommand{\ddt}{\frac{d}{dt}}
\newcommand{\bex}{\begin{example}}
\newcommand{\eex}{\end{example}}
\newcommand{\tdM}{\tilde{M}}
\newcommand{\tdN}{\tilde{N}}
\newcommand{\Sol}{\ni\ti{Solution. }}


\renewcommand{\div}{\text{div}}
\newcommand{\ind}{\mathbf{1}}

\title{MATH 53 Review for Midterm 1}
\author{Saran Ahuja}
                                       
\begin{document}
\maketitle

\ni In the past three weeks, we have covered many cases of ODE of the form
$$ M(x,y)+N(x,y)y' = 0 $$
\vspace{0.2in}
\subsection*{Linear First Order ODE}
If $N(x,y)=1, M(x,y) = a(x)y-b(x)$, then the ODE is called linear (since $M$ is linear in $y$). In other words, the ODE is of the form
$$ y' + a(x)y = b(x) $$
To solve this equation, we do the following step
\bi
\item multiply by integrating factor $\mu(x) = e^{\int a(x) dx}$. In the case of constant coefficient, $a(x) = a$, then the integrating factor is simply $e^{ax}$
\item Then we will have $\frac{d}{dx}(e^{\int a(x)dx}y) = e^{\int a(x)dx} b(x) $. Integrating both side and rearrange the term, and we get the solution. Don't forget the CONSTANT when integrating.
\item If given the initial condition, plug it in to find the constant.
\ei

\vspace{0.2in}
\subsection*{Separable Equation}
If  $N$ is a function of $y$ only and $M$ is a function of $x$ only, then the ODE is called separable. To see if the ODE is separable, one systematic way of checking that is to rearrange the terms by moving everything but $y'$ to the right so that you have
$$ y' = f(y,x)$$
If you can separate the RHS by writing it as a PRODUCT of two functions, one is a function of $y$ only, the other is a function of $x$ only, 
$$ y' = f(y,x) = g(y)h(x) $$
then it is separable. If it is separable, then we can solve the ODE by following these steps
\bi
\item Rewrite the ODE as 
$$  ( .. y .. )dy = (.. x .. )dx $$
\item Take the integral both side then we get the \textit{implicit} relationship between $y$ and $x$. 
\item If given the initial condition, plugging in to find the constant
\item If possible, rearrange terms to get the \textit{explicit} answer, i.e.. $y(x) = \dots$. 
\ei

\vspace{0.2in}
\subsection*{Exact equation}
If $M,N$ satisifes
$$ M_{y} = N_{x}$$
then the ODE is called exact and we can solve the ODE as follows;
\bi
\item First check that $M_{y} = N_{x}$. If it's not, we can try to make it exact (see next topic)
\item Write $ H_{x}(x,y)  = M(x,y), \quad H_{y}(x,y) = N(x,y) $ 
\item Choose one of the two equations above that looks easier to work with, take the integral (suppose I start with the second equations) to get
$$ H(x,y) = \int N(x,y)dy + g(x) $$
\item  Recall that our goal is to find $H$ and we almost know what $H$ is except that we don't know what $g$ is. We only know that it's just a function of $x$. To find $g$, we take the partial derivative and match that with the other condition. In this case, we take partial with respect to $x$. From this we get $g'(x) = \dots$, then we can solve for $g(x)$. \\ 
\item We now know what $H(x,y)$ is and the solution is given by
$$ H(x,y) = C $$
\item The last step is the same as usual. Plugging in the initial condition, if given, to find $C$, then rearrange the terms to write $y(x) = \dots$ if possible.
\ei

\vspace{0.2in}
\subsection*{Exact equation with integrating factor}
Suppose the ODE is not exact, we can try to multiply by a function to make it exact by the following step.
\bi
\item Assuming $M_{y} \neq N_{x}$ (otherwise, follow steps in the previous section) ,then we check if 
$$\frac{M_{y}(x,y)-N_{x}(x,y)}{N(x,y)} \text{ is a function of only }x$$
If it is, let $a(x) = \frac{M_{y}(x,y)-N_{x}(x,y)}{N(x,y)}$, then the integrating factor is
$$ \mu(x) = e^{\int a(x) dx } $$
\item If $ \frac{M_{y}(x,y)-N_{x}(x,y)}{N(x,y)}$ is not a function of $x$ only but has some $y$ in there. We can try to look for an integrating factor which is a function of $y$ only. To do that, we check that
$$\frac{N_{x}(x,y)-M_{y}(x,y)}{M(x,y)} \text{ is a function of only }y$$
If it is, let $b(y) = \frac{N_{x}(x,y)-M_{y}(x,y)}{M(x,y)}$, then the integrating factor is
$$ \mu(y) = e^{\int b(y) dy } $$
\item Multiply the original ODE by $\mu(x)$ (or $\mu(y)$)
\item Now the equation becomes exact and we can solve it in the same way as described in the previous section.
\ei 
\ni\tf{Remark} See the section note for detail how to get the integrating factor and why the conditions appear this way. I don't recommend memorizing all these conditions. You should understand where they come from and should be able to derive them. 






 






\end{document}  