\documentclass[10pt]{amsart}
\usepackage{geometry}                % See geometry.pdf to learn the layout options. There are lots.
\geometry{letterpaper}                   % ... or a4paper or a5paper or ... 
%\geometry{landscape}                % Activate for for rotated page geometry
%\usepackage[parfill]{parskip}    % Activate to begin paragraphs with an empty line rather than an indent
\usepackage{fullpage} %to reduce margin
\usepackage{graphicx}
\usepackage{amssymb}
\usepackage{amsthm}
\usepackage{amsmath}
\usepackage{epstopdf}
\usepackage{color}
\usepackage{framed} % or, "mdframed"
%\usepackage[framed]{ntheorem}
%\newframedtheorem{theo}{Theorem}
\newtheorem{thm}{Theorem}
\newtheorem{example}{Example}


\DeclareGraphicsRule{.tif}{png}{.png}{`convert #1 `dirname #1`/`basename #1 .tif`.png}


\newcommand{\bi}{\begin{itemize}}
\renewcommand{\i}{\item}
\newcommand{\ei}{\end{itemize}}
\renewcommand{\ni}{\noindent}
\newcommand{\tf}{\textbf}
\newcommand{\ti}{\textit}
\newcommand{\ld}{\lambda}
\renewcommand{\l}{\left}
\renewcommand{\r}{\right}
\newcommand{\la}{\langle}
\newcommand{\ra}{\rangle}
\newcommand{\tsg}{\tilde{\sigma}}
\newcommand{\sg}{\sigma}
\newcommand{\mc}{\mathcal}
\newcommand{\R}{\mathbb{R}}
\newcommand{\Rd}{\mathbb{R}^{d}}
\newcommand{\Rtd}{\mathbb{R}^{2d}}
\newcommand{\eps}{\varepsilon}

\newcommand{\cA}{\mathcal{A}}
\newcommand{\cH}{\mathcal{H}}
\newcommand{\cP}{\mathcal{P}}
\newcommand{\cL}{\mathcal{L}}
\newcommand{\PoR}{\mathcal{P}_{1}(\mathbb{R})}
\newcommand{\tdW}{d\tilde{W}}
\newcommand{\tZ}{\tilde{Z}}
\newcommand{\halpha}{\hat{\alpha}}
\newcommand{\hX}{\hat{X}}
\newcommand{\hY}{\hat{Y}}
\newcommand{\hZ}{\hat{Z}}
\newcommand{\htZ}{\hat{\tilde{Z}}}
\newcommand{\hm}{\hat{m}}
\renewcommand{\d}[1]{\partial_{#1}}
\newcommand{\Ve}{V^{\eps}}
\newcommand{\ue}{u^{\eps}}
\newcommand{\me}{m^{\eps}}
\newcommand{\Xie}{X^{i,\eps}}
\newcommand{\aie}{\alpha^{i,\eps}}
\renewcommand{\ae}{\alpha^{\eps}}
\newcommand{\Vo}{V^{0}}
\newcommand{\uo}{u^{0}}
\newcommand{\mo}{m^{0}}
\newcommand{\Xio}{X^{i,0}}
\newcommand{\aio}{\alpha^{i,0}}
\newcommand{\ao}{\alpha^{0}}
\newcommand{\du}{ \delta^{u,\eps}}
\newcommand{\dm}{ \delta^{m,\eps}}
\newcommand{\duo}{ \delta^{u}}
\newcommand{\dmo}{ \delta^{m}}
\newcommand{\mC}{\mathcal}

\newcommand{\imply}{ \quad \Rightarrow \quad}
\newcommand{\bex}{\begin{example}}
\newcommand{\eex}{\end{example}}
\renewcommand{\div}{\text{div}}
\newcommand{\ind}{\mathbf{1}}

\title{MATH 53 Note: 04/11/2013}
\author{Saran Ahuja}
                                       
\begin{document}
\maketitle

\bex Find the explicit solution of
$$ y' = \frac{e^{-x}}{2y+3},\quad y(0) = -1 $$
and determine the maximal interval of existence
\eex 
\ni\ti{Solution. } Separate the variable,
$$  (2y+3)dy = e^{-x}dx \imply \int (2y+3)dy = \int e^{-x}dx + C \imply y^{2}+3y = -e^{-x}+C $$
Plugging in $y(0) = -1$,
$$ 1-3 = -1+C \imply C = -1 $$
so the solution $y(x)$ satisfies
$$ y^{2}(x) + 3y(x) = -e^{-x}-1 $$
By the quadratic formula (or completing the square), we get
$$ y(x) = \frac{-3 \pm \sqrt{9-4(e^{-x}+1)}}{2} = \frac{-3 \pm \sqrt{5-4e^{-x}}}{2} $$
But $y(0)=-1$, so 
$$ y(x) = \frac{-3 + \sqrt{5-4e^{-x}}}{2} $$
The solution is well-defined only when 
$$5-4e^{-x} \geq 0 \imply x \geq -\ln(5/4)$$
Thus, the maximal interval of existence is (-$\ln(5/4),\infty$).
 
\vspace{0.2in}
\bex Find the explicit solution of
$$ ty' + y^{2} = 0, \quad y(1) = 1$$
and determine the maximal interval of existence
\eex 
\ni\ti{Solution. } Separate the variable
$$ -\frac{1}{y^{2}}dy = \frac{1}{t}dt \imply  \int-\frac{1}{y^{2}}dy = \int \frac{1}{t}dt + C \imply \frac{1}{y(t)} = \ln |t| + C $$
Plugging in $y(1) = 1$,
$$ 1 = 0 + C  \imply C =1 $$
so the solution $y(t)$ is
$$ y(t) = \frac{1}{\ln |t| + 1 } $$
The solution will not be defined when $\ln|t| = -1$ or $t = 0$. The first case happens when $|t| = e^{-1}$ or $t = \pm e^{-1}$. Combining all cases, the solution will not be defined when $t = -e^{-1},0,e^{-1}$. Since the initial is given at $t=1$, the maximum interval of existence is $(e^{-1},\infty)$.

\vspace{0.2in}
\bex Find the explicit solution of 
$$ y'(t) - e^{-y}\cos t = 0, \quad y(0) = y_{0} $$
For which $y_{0}$ is the solution $y(t)$ defined for all $-\infty < t < \infty$.
\eex 
\ni\ti{Solution} Separate the variable 
$$ y'(t) = e^{-y}\cos t \imply e^{y}dy = \cos t dt \imply \int e^{y}dy = \int \cos t dt $$
Thus,
$$ e^{y(t)} = \sin t + C $$
Plugging $y(0) = y_{0}$ (think of $y_{0}$ as some number given to you), 
$$ e^{y_{0}} = 0 + C \imply C = e^{y_{0}} $$
so the solution is
$$ e^{y(t)} = \sin t + e^{y_{0}} \imply y(t) = \ln ( \sin t + e^{y_{0}} ) $$
For solution to be defined everywhere, the term inside the $\ln $ function must be positive, that is
$$  e^{y_{0}} > -\sin t, \text{for all }t $$
$- \sin t$ has a maximum value of $1$, so we must have 
$$  e^{y_{0}}  > 1 \imply y_{0} > 0 $$

\vspace{0.2in}
\bex Consider the ODE
$$ y'  = y^{2}-y^{4} $$
Determine all equilibrium points and its stability properties.
\eex
\ni\ti{Solution} The equilibrium is where $y'(t) = 0$, so we solve
$$ y^{2} - y^{4} = 0 \imply y^{2}(1-y)(1+y) = 0 \imply y = -1,0,1 $$
Drawing the directional field or just by looking at the sign of $y' (= y^{2}-y^{4})$ when $y$ is in $(-\infty, -1), (-1,0), (0,1), (1,\infty)$. The sign is -,+,+,- respectively, so we deduce that -1 is unstable, 1 is asymptotically stable. 0 is called semistable. Note that we can solve this equation by using separation of variable and partial fraction (try it!). The solution cannot be described explicitly, but yet we can tell a lot about the solution by looking at the directional field plot.
 
 
% \begin{example} Identify if these equations are linear first order ODE and/or separable. 
%\begin{enumerate}
%\item $u' - u^{2} = t\ln u $
%\item $u' =tu+1$
%\item$u' =  tu(u-2)$
%\item$u' = tu(u-2) + 1$
%\item$u' =( t+1)(u(u-2)+1)$
%\end{enumerate}
%(1) is linear ODE and separable, (2) is linear ODE but not separable. (3) is non-linear but separable. (4) is neither linear nor separable. (5) is non-linear but separable. 
%\end{example} 
 


 






\end{document}  