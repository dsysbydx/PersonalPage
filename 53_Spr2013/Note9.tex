\documentclass[10pt]{amsart}
\usepackage{geometry}                % See geometry.pdf to learn the layout options. There are lots.
\geometry{letterpaper}                   % ... or a4paper or a5paper or ... 
%\geometry{landscape}                % Activate for for rotated page geometry
%\usepackage[parfill]{parskip}    % Activate to begin paragraphs with an empty line rather than an indent
\usepackage{fullpage} %to reduce margin
\usepackage{graphicx}
\usepackage{amssymb}
\usepackage{amsthm}
\usepackage{amsmath}
\usepackage{epstopdf}
\usepackage{color}
\usepackage{framed} % or, "mdframed"
%\usepackage[framed]{ntheorem}
%\newframedtheorem{theo}{Theorem}
\newtheorem{thm}{Theorem}
\newtheorem{example}{Example}


\DeclareGraphicsRule{.tif}{png}{.png}{`convert #1 `dirname #1`/`basename #1 .tif`.png}


\newcommand{\bi}{\begin{itemize}}
\renewcommand{\i}{\item}
\newcommand{\ei}{\end{itemize}}
\renewcommand{\ni}{\noindent}
\newcommand{\tf}{\textbf}
\newcommand{\ti}{\textit}
\newcommand{\ld}{\lambda}
\renewcommand{\l}{\left}
\renewcommand{\r}{\right}
\newcommand{\la}{\langle}
\newcommand{\ra}{\rangle}
\newcommand{\tsg}{\tilde{\sigma}}
\newcommand{\sg}{\sigma}
\newcommand{\mc}{\mathcal}
\newcommand{\R}{\mathbb{R}}
\newcommand{\Rd}{\mathbb{R}^{d}}
\newcommand{\Rtd}{\mathbb{R}^{2d}}
\newcommand{\eps}{\varepsilon}

\newcommand{\cA}{\mathcal{A}}
\newcommand{\cH}{\mathcal{H}}
\newcommand{\cP}{\mathcal{P}}
\newcommand{\cL}{\mathcal{L}}
\newcommand{\PoR}{\mathcal{P}_{1}(\mathbb{R})}
\newcommand{\tdW}{d\tilde{W}}
\newcommand{\tZ}{\tilde{Z}}
\newcommand{\halpha}{\hat{\alpha}}
\newcommand{\hX}{\hat{X}}
\newcommand{\hY}{\hat{Y}}
\newcommand{\hZ}{\hat{Z}}
\newcommand{\htZ}{\hat{\tilde{Z}}}
\newcommand{\hm}{\hat{m}}
\renewcommand{\d}[1]{\partial_{#1}}
\newcommand{\Ve}{V^{\eps}}
\newcommand{\ue}{u^{\eps}}
\newcommand{\me}{m^{\eps}}
\newcommand{\Xie}{X^{i,\eps}}
\newcommand{\aie}{\alpha^{i,\eps}}
\renewcommand{\ae}{\alpha^{\eps}}
\newcommand{\Vo}{V^{0}}
\newcommand{\uo}{u^{0}}
\newcommand{\mo}{m^{0}}
\newcommand{\Xio}{X^{i,0}}
\newcommand{\aio}{\alpha^{i,0}}
\newcommand{\ao}{\alpha^{0}}
\newcommand{\du}{ \delta^{u,\eps}}
\newcommand{\dm}{ \delta^{m,\eps}}
\newcommand{\duo}{ \delta^{u}}
\newcommand{\dmo}{ \delta^{m}}
\newcommand{\mC}{\mathcal}
\newcommand{\imply}{ \quad \Rightarrow \quad}
\newcommand{\ddt}{\frac{d}{dt}}
\newcommand{\bex}{\begin{example}}
\newcommand{\eex}{\end{example}}
\newcommand{\tdM}{\tilde{M}}
\newcommand{\tdN}{\tilde{N}}
\newcommand{\Sol}{\ni\ti{Solution. }}

\renewcommand{\v}{\vec{v}}
\newcommand{\vb}[1]{\vec{\textbf{#1}}}
\newcommand{\vecx}{\vec{\mathbf{x}}}
\newcommand{\hx}{\hat{\mathbf{x}}}


\renewcommand{\div}{\text{div}}
\newcommand{\ind}{\mathbf{1}}
\renewcommand{\Vec}[2]{\l[ \begin{tabular}{c} $#1$ \\ $#2$ \end{tabular} \r]}
\newcommand{\Mat}[4]{\l[ \begin{tabular}{cc} $#1$ &$#2$ \\ $#3$ &$#4$ \end{tabular} \r]}

\title{MATH 53 Note: 04/30/2013}
\author{Saran Ahuja}
                                       
\begin{document}
\maketitle
\section{Solving $\vecx'(t) = A\vecx(t)$}

Recall that our strategy for solving $\vecx'(t) = A\vecx(t)$ is as follows;

\begin{itemize}
	\item First, we look for two ``independent'' solutions, called them $ \vecx_{1}(t),\vecx_{2}(t)$.
	\item Form a general solutions by taking a linear combination of those two solutions, that is, the general solution is
	$$  C_{1}\vecx_{1}(t)+C_{2}\vecx_{2}(t)  $$
	\item Plug in the initial condition (if given) to find $C_{1},C_{2}$. 
\end{itemize}
Let's look at the one dimensional problem to get some idea of how to find a solution for 2D problem. The analogue of this equation in 1D is simply
$$ u'(t) = cu(t) $$
and a solution is given by
$$ u(t) = e^{ct} $$
From that, we will first look for a solution of the form
$$ \vecx(t) = e^{\ld t}\Vec{a}{b} $$
We want to find out which $\ld, a,b$ will make $\vecx(t)$ a solution. By plugging this back in $\vecx'(t) = A\vecx(t)$, we get that
$$ \ld e^{\ld t}\Vec{a}{b} = A e^{\ld t}\Vec{a}{b} \imply A\Vec{a}{b}  = \ld  \Vec{a}{b}  $$
That is, we have just shown that \textbf{ if $\ld$ is an eigenvalue and $\Vec{a}{b} $ is a corresponding eigenvector of matrix $A$, then
$$   \vecx(t) = e^{\ld t}\Vec{a}{b}  $$
is a solution to  $\vecx'(t) = A\vecx(t)$.}

The recipe for finding two solutions will depend on eigenvalues of matrix $A$, which can be divided into three cases;
\begin{itemize}
	\item $A$ has two different real eigenvalues $\lambda_{1} \neq \lambda_{2}$
	\item $A$ has complex eigenvalues $\lambda_{1} = \alpha+\beta i , \lambda_{2}= \alpha-\beta i $
	\item $A$ has repeated eigenvalues $\lambda_{1}=\lambda_{2}$
\end{itemize}

\section{Case I: two different real eigenvalues $\ld_{1} \neq \ld_{2}$ }
Assume $A$ has two different real-valued eigenvalues $\lambda_{1} \neq \ld_{2}$. Let $\vec{v}_{1},\vec{v}_{2}$ denote the corresponding eigenvectors. In this case, we have two solution right away
$$ \vecx_{1}(t) = e^{\ld_{1}t}\vec{v}_{1},\quad \vecx_{2}(t) = e^{\ld_{2}t}\vec{v}_{2} $$
so the general solution is given by
$$ \vecx(t) = C_{1}e^{\ld_{1}t}\vec{v}_{1} + C_{2}e^{\ld_{2}t}\vec{v}_{2} $$

\begin{example} Solve system of ODE
$$ x'(t) = 3x(t)+2y(t), y'(t) = x(t)+2y(t), x(0)=0,y(0)=-3 $$
\end{example}

\Sol We first write an equation in a matrix form
$$ \vb{x}'(t) = \Mat{3}{2}{1}{2} \vb{x}(t),\qquad \vb{x}(0) = \Vec{0}{-3} $$
We then find an eigenvalue of $A =  \Mat{3}{2}{1}{2}$,
$$ \ld I - A = \Mat{\ld - 3}{-2}{-1}{\ld - 2} \imply (\ld-3)(\ld-2) - 2 = 0 \imply \ld^{2}-5\ld +4 = 0 \imply \ld = 4,1 $$
Start with $\ld_{1} = 4$,
$$ 4I - A = \Mat{1}{-2}{-1}{2} \imply \vec{v}_{1} = \Vec{2}{1} $$
We have found one solution to the system of ODE,
$$ \vb{x}_{1}(t) = e^{4t}\Vec{2}{1} $$
Next, we work with $\ld_{2}= 1$,
$$ 1I - A = \Mat{-2}{-2}{-1}{-1} \imply \vec{v}_{1} = \Vec{1}{-1} $$
We have found a second solution to the system of ODE,
$$ \vb{x}_{2}(t) = e^{t}\Vec{1}{-1} $$
Now that we have found two solutions, the general solution is simply
$$ \vb{x}(t) = C_{1} e^{4t}\Vec{2}{1} +C_{2} e^{t}\Vec{1}{-1} $$
Plugging in initial condition $t=0$ yields
$$ \Vec{0}{-3} =  \vb{x}(0) = C_{1}\Vec{2}{1} +C_{2} \Vec{1}{-1}  = \Vec{2C_{1}+C_{2}}{C_{1}-C_{2}} $$
Solving this, we get $C_{1}= -1, C_{2}=2$, thus the solution to the IVP is
$$ \vb{x}(t) = -e^{4t}\Vec{2}{1} +2 e^{t}\Vec{1}{-1} = \Vec{ -2e^{-4t}+2e^{t}}{-e^{4t}-2e^{t}}  $$





 






\end{document}  