\documentclass[10pt]{amsart}
\usepackage{geometry}                % See geometry.pdf to learn the layout options. There are lots.
\geometry{letterpaper}                   % ... or a4paper or a5paper or ... 
%\geometry{landscape}                % Activate for for rotated page geometry
%\usepackage[parfill]{parskip}    % Activate to begin paragraphs with an empty line rather than an indent
\usepackage{fullpage} %to reduce margin
\usepackage{graphicx}
\usepackage{amssymb}
\usepackage{amsthm}
\usepackage{amsmath}
\usepackage{epstopdf}
\usepackage{color}
\usepackage{framed} % or, "mdframed"
%\usepackage[framed]{ntheorem}
%\newframedtheorem{theo}{Theorem}
\newtheorem{thm}{Theorem}
\newtheorem{example}{Example}


\DeclareGraphicsRule{.tif}{png}{.png}{`convert #1 `dirname #1`/`basename #1 .tif`.png}


\newcommand{\bi}{\begin{itemize}}
\renewcommand{\i}{\item}
\newcommand{\ei}{\end{itemize}}
\renewcommand{\ni}{\noindent}
\newcommand{\tf}{\textbf}
\newcommand{\ti}{\textit}
\newcommand{\ld}{\lambda}
\renewcommand{\l}{\left}
\renewcommand{\r}{\right}
\newcommand{\la}{\langle}
\newcommand{\ra}{\rangle}
\newcommand{\tsg}{\tilde{\sigma}}
\newcommand{\sg}{\sigma}
\newcommand{\mc}{\mathcal}
\newcommand{\R}{\mathbb{R}}
\newcommand{\Rd}{\mathbb{R}^{d}}
\newcommand{\Rtd}{\mathbb{R}^{2d}}
\newcommand{\eps}{\varepsilon}

\newcommand{\cA}{\mathcal{A}}
\newcommand{\cH}{\mathcal{H}}
\newcommand{\cP}{\mathcal{P}}
\newcommand{\cL}{\mathcal{L}}
\newcommand{\PoR}{\mathcal{P}_{1}(\mathbb{R})}
\newcommand{\tdW}{d\tilde{W}}
\newcommand{\tZ}{\tilde{Z}}
\newcommand{\halpha}{\hat{\alpha}}
\newcommand{\hX}{\hat{X}}
\newcommand{\hY}{\hat{Y}}
\newcommand{\hZ}{\hat{Z}}
\newcommand{\htZ}{\hat{\tilde{Z}}}
\newcommand{\hm}{\hat{m}}
\renewcommand{\d}[1]{\partial_{#1}}
\newcommand{\Ve}{V^{\eps}}
\newcommand{\ue}{u^{\eps}}
\newcommand{\me}{m^{\eps}}
\newcommand{\Xie}{X^{i,\eps}}
\newcommand{\aie}{\alpha^{i,\eps}}
\renewcommand{\ae}{\alpha^{\eps}}
\newcommand{\Vo}{V^{0}}
\newcommand{\uo}{u^{0}}
\newcommand{\mo}{m^{0}}
\newcommand{\Xio}{X^{i,0}}
\newcommand{\aio}{\alpha^{i,0}}
\newcommand{\ao}{\alpha^{0}}
\newcommand{\du}{ \delta^{u,\eps}}
\newcommand{\dm}{ \delta^{m,\eps}}
\newcommand{\duo}{ \delta^{u}}
\newcommand{\dmo}{ \delta^{m}}
\newcommand{\mC}{\mathcal}
\newcommand{\imply}{ \quad \Rightarrow \quad}
\newcommand{\ddt}{\frac{d}{dt}}
\newcommand{\bex}{\begin{example}}
\newcommand{\eex}{\end{example}}


\renewcommand{\div}{\text{div}}
\newcommand{\ind}{\mathbf{1}}

\title{MATH 53 Note: 04/16/2013}
\author{Saran Ahuja}
                                       
\begin{document}
\maketitle
\section{Exact equation}

\ni Recall that the first order ODE
$$ M(x,y)+N(x,y)y' = 0 $$
is called \textit{exact} if 
$$ M_{y}(x,y) = N_{x}(x,y) $$
In this case, there exist $H(x,y)$ such that
$$ H_{x}(x,y) = M(x,y),\quad H_{y}(x,y) =N(x,y) $$
and the solution is given by
$$ H(x,y(x)) = C$$
Note that \textit{separable} equation is a special case of exact equation. When the ODE is separable, we get
$$ M_{y} = 0 = N_{x} $$
To find $H$, we start with either $H_{x} = M$ or $H_{y} = N$, then we take the integral. Since we are dealing with partial derivative, the constant is a function of another variable. Then we need the other condition to determine the constant. This might be a bit confusing for now, but let's look at some examples to see how this works.

\bex Determine if the following ODE is exact or not.
\begin{enumerate}
	\item $2y+xy' = 0$
	\item $2xy + x^{2}y' = 0$
	\item $xy' = -1-2x^{2}$
	\item $e^{x^{2}+y}(1+2x^{2})+xe^{x^{2}+y}y' = 0$
	\item $3xy+y^{2} + (x^{2}+xy)y' = 0$
	\item  $3x^{2}y+xy^{2} + (x^{3}+x^{2}y)y' = 0$ 
\end{enumerate}
\eex
\ni\ti {Answer} (1),(3),(5) are not exact, but after multiplying by integrating factors, we get (2),(4),(6) [respectively] which are exact. 

\bex Solve  
$$3x(xy-2) + (x^{3}+ 2y) y' = 0$$ 
with initial condition $y(0)=1$
\eex

\ni\ti{Solution} Note that
$$ M(x,y) = 3x(xy-2),\quad N(x,y) = x^{3}+2y $$
and
$$ M_{y} = 3x^{2} = N_{x} $$
so the ODE is exact. We want to find $H(x,y)$ such that
$$ H_{x} = 3x(xy-2) = 3x^{2}y-6x,\quad H_{y} = x^{3}+2y $$
We start with the second condition. Taking the integral with respect to $y$ gives
$$ H(x,y) = x^{3}y+y^{2}+C(x)$$
Taking derivative of this gives
$$ H_{x}(x,y) = 3x^{2}y+y^{2}+C'(x) $$
Comparing with the first condition gives
$$ C'(x) = -6x,\quad C(x) = -3x^{2} $$
So we have that 
$$ H(x,y) = x^{3}y+y^{2}-3x^{2}$$
and the general solution is given by
$$ x^{3}y(x)+y(x)^{2}-3x^{2} = C $$
Given the initial condition $y(0)=1$, we plug in $x=0,y(x)=y(0)=1$ to find that $C=1$. The solution is
$$ x^{3}y(x)+y(x)^{2}-3x^{2} = 1 $$
We can write this out explicitly using quadratic formula to get
$$ y(x) = \frac{-x^{3}\pm\sqrt{x^{6}-4(-3x^{2}-1)}}{2} $$
Checking with initial $y(0)=1$, we know it's the solution with the plus sign, so the explicit solution is
$$ y(x) = \frac{-x^{3}+\sqrt{x^{6}+12x^{2}+4}}{2}  $$

\bex Solve  
$$(\cos y + y \cos x) + (\sin x - x \sin y) y' = 0$$ 
with initial condition $y(0) = -1$
\eex

\ni\ti{Solution}  Note that
$$ M(x,y) =\cos y + y \cos x,\quad N(x,y) = \sin x - x \sin y $$
and
$$ M_{y} = -\sin y + \cos x = N_{x} $$
so the ODE is exact. We want to find $H(x,y)$ such that
$$ H_{x} = \cos y + y \cos x,\quad H_{y} = \sin x - x \sin y $$
This time, we take the first condition and take the integral with respect to $x$ and get
$$ H(x,y) = x\cos y + y\sin x + C(y) $$
Taking derivative with respect to $y$,
$$ H_{y}(x,y) = -x\sin y + \sin x + C'(y) $$
Comparing with the second condition, we get
$$ C'(y) = 0 \imply C(y) = C $$
So the general solution is
$$ H(x,y) = C \imply x\cos y + y\sin x = C $$
Plugging in initial condition $y(0)=-1$, we get
$$ C = H(0,-1) = 0 + 0 = 0 $$
So the solution is
$$  x\cos y + y\sin x = 0 $$
In this case, we cannot find an explicit formula.





 






\end{document}  