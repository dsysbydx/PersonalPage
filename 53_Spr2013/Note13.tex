\documentclass[10pt]{amsart}
\usepackage{geometry}                % See geometry.pdf to learn the layout options. There are lots.
\geometry{letterpaper}                   % ... or a4paper or a5paper or ... 
%\geometry{landscape}                % Activate for for rotated page geometry
%\usepackage[parfill]{parskip}    % Activate to begin paragraphs with an empty line rather than an indent
\usepackage{fullpage} %to reduce margin
\usepackage{graphicx}
\usepackage{amssymb}
\usepackage{amsthm}
\usepackage{amsmath}
\usepackage{epstopdf}
\usepackage{color}
\usepackage{framed} % or, "mdframed"
%\usepackage[framed]{ntheorem}
%\newframedtheorem{theo}{Theorem}
\newtheorem{thm}{Theorem}
\newtheorem{example}{Example}


\DeclareGraphicsRule{.tif}{png}{.png}{`convert #1 `dirname #1`/`basename #1 .tif`.png}


\newcommand{\bi}{\begin{itemize}}
\renewcommand{\i}{\item}
\newcommand{\ei}{\end{itemize}}
\renewcommand{\ni}{\noindent}
\newcommand{\tf}{\textbf}
\newcommand{\ti}{\textit}
\newcommand{\ld}{\lambda}
\renewcommand{\l}{\left}
\renewcommand{\r}{\right}
\newcommand{\la}{\langle}
\newcommand{\ra}{\rangle}
\newcommand{\tsg}{\tilde{\sigma}}
\newcommand{\sg}{\sigma}
\newcommand{\mc}{\mathcal}
\newcommand{\R}{\mathbb{R}}
\newcommand{\Rd}{\mathbb{R}^{d}}
\newcommand{\Rtd}{\mathbb{R}^{2d}}
\newcommand{\eps}{\varepsilon}

\newcommand{\cA}{\mathcal{A}}
\newcommand{\cH}{\mathcal{H}}
\newcommand{\cP}{\mathcal{P}}
\newcommand{\cL}{\mathcal{L}}
\newcommand{\PoR}{\mathcal{P}_{1}(\mathbb{R})}
\newcommand{\tdW}{d\tilde{W}}
\newcommand{\tZ}{\tilde{Z}}
\newcommand{\halpha}{\hat{\alpha}}
\newcommand{\hX}{\hat{X}}
\newcommand{\hY}{\hat{Y}}
\newcommand{\hZ}{\hat{Z}}
\newcommand{\htZ}{\hat{\tilde{Z}}}
\newcommand{\hm}{\hat{m}}
\renewcommand{\d}[1]{\partial_{#1}}
\newcommand{\Ve}{V^{\eps}}
\newcommand{\ue}{u^{\eps}}
\newcommand{\me}{m^{\eps}}
\newcommand{\Xie}{X^{i,\eps}}
\newcommand{\aie}{\alpha^{i,\eps}}
\renewcommand{\ae}{\alpha^{\eps}}
\newcommand{\Vo}{V^{0}}
\newcommand{\uo}{u^{0}}
\newcommand{\mo}{m^{0}}
\newcommand{\Xio}{X^{i,0}}
\newcommand{\aio}{\alpha^{i,0}}
\newcommand{\ao}{\alpha^{0}}
\newcommand{\du}{ \delta^{u,\eps}}
\newcommand{\dm}{ \delta^{m,\eps}}
\newcommand{\duo}{ \delta^{u}}
\newcommand{\dmo}{ \delta^{m}}
\newcommand{\mC}{\mathcal}
\newcommand{\imply}{ \quad \Rightarrow \quad}
\newcommand{\ddt}{\frac{d}{dt}}
\newcommand{\bex}{\begin{example}}
\newcommand{\eex}{\end{example}}
\newcommand{\tdM}{\tilde{M}}
\newcommand{\tdN}{\tilde{N}}
\newcommand{\Sol}{\ni\ti{Solution. }}

\renewcommand{\v}{\vec{\mathbf{v}}}
\newcommand{\w}{\vec{\mathbf{w}}}
\newcommand{\x}{\vec{\mathbf{x}}}
\newcommand{\X}{\mathbf{X}}
\renewcommand{\b}{\vec{\mathbf{b}}}

\newcommand{\vb}[1]{\vec{\textbf{#1}}}
\newcommand{\vecx}{\vec{\mathbf{x}}}
\newcommand{\hx}{\hat{\mathbf{x}}}


\renewcommand{\div}{\text{div}}
\newcommand{\ind}{\mathbf{1}}
\newcommand{\TwoLine}[2]{ \begin{tabular}{c} $#1$ \\ $#2$ \end{tabular}}
\renewcommand{\Vec}[2]{\l[ \begin{tabular}{c} $#1$ \\ $#2$ \end{tabular} \r]}
\newcommand{\Mat}[4]{\l[ \begin{tabular}{cc} $#1$ &$#2$ \\ $#3$ &$#4$ \end{tabular} \r]}

\title{MATH 53 Note: 05/14/2013}
\author{Saran Ahuja}
                                       
\begin{document}
\maketitle

\section{Second order linear ODE with constant coefficients: homogeneous}
In this note, we will discuss how to solve an equation of the form
$$ ay''(t)+by'(t)+cy(t) = 0 $$
Note that this can be turned into the system of ODE by letting
$$ \x(t) = \Vec{y(t)}{y'(t)} $$
Then
$$ \x'(t) = \Mat{0}{1}{-c/a}{-b/a} \x(t) $$
So we can see the solution $y(t)$ should take the form $e^{\ld t}$. By plugging that in and solve for $\ld$, we can find solutions to the second order ODE as follows;
\begin{enumerate}
	\item Find the roots of characteristic equation $a\ld^{2}+b\ld + c = 0$ 
	\item The roots fall into one of the three cases;
	\begin{itemize}
		\item $\ld_{1}\neq \ld_{2}$ and are both real. Then the general solution is
		$$ y(t) = C_{1}e^{\ld_{1} t}+C_{2}e^{\ld_{2}t} $$
		\item $\ld_1 = \alpha+\beta i, \ld_{2} = \alpha-\beta i$. The general solution is
		$$ y(t) = C_{1}e^{\alpha t}\cos \beta t + C_{2}e^{\alpha t}\sin \beta t $$
		\item $\ld_{1} = \ld_{2} = \ld $ (must be real). The general solution is
		$$ y(t) = C_{1}e^{\ld t}+C_{2}te^{\ld t} $$
	\end{itemize}
	\item Find $y'(t)$ if the initial condition involves $y'(t_{0})$.
	\item Plugging in initial conditions to find $C_{1},C_{2}$
\end{enumerate}

\vspace{0.2in}
\begin{example} Solve $y'' - 5y' +6y = 0$ with initial condition $y(0) =1 , y'(0) =1 $
\end{example}

\Sol the characteristic equation is
$$\ld^{2}-5\ld+6 = 0 \imply (\ld-3)(\ld-2) = 0 \imply \ld = 2,3 $$
The general solution is then given by
$$ y(t) = C_{1}e^{2t}+C_{2}e^{3t} $$
We first find its derivative so we can plug in the initial condition,
$$ y'(t) = 2C_{1}e^{2t}+3C_{2}e^{3t} $$
Plugging in $t=0$ to both $y(t),y'(t)$, 
$$ \TwoLine{C_{1}+C_{2} = 1 }{2C_{1}+3C_{2}= 1} \imply C_{1}=2,C_{2}= -1$$
So the solution to the IVP is
$$ y(t) = 2e^{2t}-e^{3t} $$ 

\vspace{0.2in}
\begin{example} Solve $y'' - 4y' +5y = 0$ with initial condition $y(0) =3 , y'(0) = 2$
\end{example}

\Sol the characteristic equation is
$$\ld^{2}-4\ld+5 = 0 \imply (\ld-2)^{2} = -1 \imply \ld = 2\pm i $$
The general solution is then given by
$$ y(t) = C_{1}e^{2t}\cos t+C_{2}e^{2t} \sin t$$
We first find its derivative so we can plug in the initial condition,
$$ y'(t) = 2C_{1}e^{2t}\cos t-C_{1}e^{2t}\sin t+ 2C_{2}e^{2t} \sin t +C_{2}e^{2t} \cos t $$
Plugging in $t=0$ to both $y(t),y'(t)$, 
$$ \TwoLine{C_{1} = 3 }{2C_{1}+C_{2}= 2} \imply C_{1}=3,C_{2}= -4$$
So the solution to the IVP is
$$ y(t) =3e^{2t}\cos t-4e^{2t} \sin t$$ 

\vspace{0.2in}
\begin{example} Solve $y'' +6y' +9y = 0$ with initial condition $y(0) = 2 , y'(0) = -5 $
\end{example}

\Sol the characteristic equation is
$$\ld^{2}+6\ld+9 = 0 \imply (\ld+3)^{2} = 0 \imply \ld = -3 $$
The general solution is then given by
$$ y(t) = C_{1}e^{-3t}+C_{2}te^{-3t} $$
We first find its derivative so we can plug in the initial condition,
$$ y'(t) = -3C_{1}e^{2t}-3C_{2}te^{-3t}+C_{2}e^{-3t} $$
Plugging in $t=0$ to both $y(t),y'(t)$, 
$$ \TwoLine{C_{1} = 2}{-3C_{1}+C_{2}= -5} \imply C_{1}=2,C_{2}= 1$$
So the solution to the IVP is
$$ y(t) = 2e^{-3t}+te^{-3t} $$







\end{document}  